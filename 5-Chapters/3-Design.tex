%----------------------------------------------------------------------
% 設計方法
%----------------------------------------------------------------------

\chapter{設計方法\small{(如何使用equation和algorithm)}}\label{chap:design}
% Write down your content here
第\ref{chap:intro}章。儘管設計方法看似不顯眼,卻佔據了我的腦海。如果別人做得到,那我也可以做到。我們不得不相信,若無法徹底理解設計方法,恐怕會是人類的一大遺憾。我們都很清楚,這是個嚴謹的議題。

章節\ref{sec:2-spec}。\ref{equ:equlabel}。我們不得不面對一個非常尷尬的事實,那就是問題的關鍵看似不明確,但想必在諸位心中已有了明確的答案。對於設計方法,我們不能不去想,卻也不能走火入魔。裡根曾講過,每一個人都應該為自己的失敗負責。我國社會的結構,是由家庭、教會、學校、左鄰右舍、政府,以及娛樂場所等交織而成的。在所有與孩子發展有關的因素中,都包含著一種角色的模仿,而影響角色模仿最大的,便是為人父母者。這不禁令我重新仔細的思考。設計方法究竟是怎麼樣的存在,始終是個謎題。由於,培根在不經意間這樣說過,無論你怎樣地表示憤怒,都不要做出任何無法挽回的事來。

\begin{equation}\label{equ:equlabel}
    \max \sum_{p, i, s}{V_{p, i} \cdot X_{p, i, s}} \ni \sum_{p, i}{W_{p, i} \cdot X_{p, i, s}} \le C_s
\end{equation}

\begin{algorithm}[htbp]
    \SetAlgoNoLine

    \caption{演算法A}
    \label{algo:algoexample}

    \Input{
        長度為$n$的序列$S$
    }

    \Output{
        序列$S$的總和
    }

    % 這是在Algorithm加一條線R
    \AlgoHRule

    $i \gets 0$\;
    $x \gets 0$\;
    \;
    \While{$i < n$}
    {
        $x \gets x + S[i]$\;
        $i  \gets i + 1$\;
    }
    \BlankLine
    \Return $x$\;
\end{algorithm}

章節\ref{sec:1-motivation}。演算法\ref{algo:algoexample}。希望大家能從這段話中有所收穫。對設計方法進行深入研究,在現今時代已經無法避免了。當前最急迫的事,想必就是釐清疑惑了。帶著這些問題,我們一起來審視設計方法。在人生的歷程中,設計方法的出現是必然的。本人也是經過了深思熟慮,在每個日日夜夜思考這個問題。設計方法絕對是史無前例的。回過神才發現,思考設計方法的存在意義,已讓我廢寢忘食。
