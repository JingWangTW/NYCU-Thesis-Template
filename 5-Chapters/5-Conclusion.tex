%----------------------------------------------------------------------
% 結論與未來展望
%----------------------------------------------------------------------

\chapter{結論與未來展望\small{(如何使用footnote)}}\label{chap:conclusion}

\section{研究成果}

表\ref{tab:tabexample6}。表\ref{tab:tabexample7}。表\ref{tab:tabexample8}。一個註腳\footnote{這是一個正常文字中的footnote}。在人生的歷程中,研究成果的出現是必然的。做好研究成果這件事,可以說已經成為了全民運動。領悟其中的道理也不是那麼的困難。對於一般人來說,研究成果究竟象徵著什麼呢?總而言之,我們普遍認為,若能理解透徹核心原理,對其就有了一定的了解程度。

\section{未來展望}

\begin{table}[ht]
    \centering
    \renewcommand{\arraystretch}{1.2}

    \begin{tabular}{ c | c | c | c | C{10em}}
        $\curlywedge $                         & \multicolumn{2}{c|}{$\curlyvee  $} & \multicolumn{2}{c}{$\Cap  $}                                     \\\hline
        $\Cup $                                & $\bot  $                           & $\top  $                     & $\doublebarwedge  $ & $\lhd  $    \\ \hline\hline
        這邊有第一個表格footnote \footnotemark & $\unlhd  $                         & $\bigtriangledown  $         & $\triangleq  $      & $\circeq  $ \\\hline
    \end{tabular}

    \renewcommand{\arraystretch}{1}

    \caption{第一種表格footnote}
    \label{tab:tabexample6}
\end{table}
\footnotetext{第一種表格footnote,這種方法一個table只能有一個footnote,而且要寫在2個地方。}

\begin{table}[ht]
    \centering
    \renewcommand{\arraystretch}{1.2}

    \begin{tabular}{ c | c | c | c | c}
        \multirow{2}{*}{$\bumpeq $}                                                                                & \multicolumn{2}{c|}{$\therefore \because  $} & \multicolumn{2}{c}{$\eqcirc \neq  $}                                                                                               \\\cline{2-5}
                                                                                                                   & $\leqq \geqq  $                              & $\leqslant \geqslant  $               & $\lessapprox \gtrapprox  $                           & $\lll \ggg $                        \\ \hline\hline
        這邊有第二個表格footnote \tablefootnote{這邊有另外一種table footnote,}                                    & $\lessdot \gtrdot  $                         & $\lesssim \gtrsim  $                  & $\eqslantless \eqslantgtr  $                         & $\precsim \succsim  $               \\\hline
        這邊有第三個表格footnote \tablefootnote{使用\textbackslash tablefootnote就可以在table內產生多組footnote,} & $\precapprox \succapprox   $                 & $\Subset \Supset   $                  & $\subseteqq \supseteqq   $                           & $\nu  $                             \\\hline
        這邊有第四個表格footnote \tablefootnote{但是也會讓你的table部分的code變得有點亂。}                         & $\preccurlyeq \succcurlyeq   $               & $\backepsilon \between \pitchfork   $ & $\ntrianglelefteq \ntriangleleft \ntriangleright   $ & $\Longleftarrow \Longrightarrow   $ \\\hline
    \end{tabular}

    \renewcommand{\arraystretch}{1}

    \caption{第二種表格footnote}
    \label{tab:tabexample7}
\end{table}

\begin{table}[ht]
    \centering
    \renewcommand{\arraystretch}{1.2}

    \begin{tabular}{ c | c | c | c | C{10em}}
        $\rightleftharpoons \leftrightharpoons $ & \multicolumn{2}{c|}{$\longleftrightarrow  $} & \multicolumn{2}{c}{$\longmapsto  $}                                                                                                                                                                                 \\\hline
        $\leadsto $                              & $\nearrow \searrow \swarrow \nwarrow  $      & $\leftleftarrows \rightrightarrows \leftrightarrows \rightleftarrows  $ & $\upuparrows \downdownarrows \twoheadleftarrow \twoheadrightarrow  $ & $\Lleftarrow \Rrightarrow \rightsquigarrow \leftrightsquigarrow  $ \\ \hline\hline
    \end{tabular}

    \renewcommand{\arraystretch}{1}

    \caption[這一格會出現在表目錄中]{這一格會出現在內文中,所以可以加入footnote\footnotemark}
    \label{tab:tabexample8}
\end{table}
\footnotetext{{這是一個出現在caption中的footnote}}